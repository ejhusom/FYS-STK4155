\section{Introduction}\label{sec:Introduction}
In machine learning, classification is a way of dealing with problems in which we wish to indentify which predefined class a set of observations can be categorized as. Typical applications of such methods are for example diagnostics in the field of medicine, where based upon the symptoms of the patient, one can use a trained model to classify what kind of illness or disease the patient has, or it can be applied to training computers to recognize patterns or features in images, such as hand written digits or letters, so that they can automatically be classified. A commonly used method for classification is Logistic regression, which aims to model the probability that a set of observations belongs to a specific class or leads to an outcome. Another method in machine learning which is often used in classification is Artificial neural networks, which is a powerful tool that can adapt to many different problems. This method is inspired by biological neurons, and can be scaled to model both simple and very complex phenomenons. Even though a neural network can be very versatile and give precise predictions, it is harder to obtain any insights about the relation between the features and the targets of the data set, compared to more traditional methods like logistic regression. Artificial neural networks are very adatable, so they are not only used for classification, but can also work with regression problems.

In the first part of this project, we apply both of these methods on a data set from a bank in Taiwan, containing the information of credit card holders and whether they default on their payments or not, and aim to train our models in predicting future defaults. In our study, we have compaired our results with those of a recent research article\cite{YehLien} that has looked at different classification methods applied to the same data set. In the second part, we use neural networks on a regression problem, and compare its performance with linear regression methods, which we have studied in an earlier project\cite{project1}. An other aspect of this project is to explore the different ways of preprocessing the data which we are given, as this can have a big impact on how well the models perform. 

In this report, we include the background theory of the methods we have used in the project, as well as a short review of how we have implemented the methods. Our findings are presented in the results section, followed by a discussion of the results and lastly a conclusion.

