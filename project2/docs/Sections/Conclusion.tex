\section{Conclusion}\label{sec:Conclusion}
In this project we study the performance of two classification methods, logistic regression and artificial neural networks, and compare their predictive performances. The prediction accuracy of the two methods have been found to be similar, but when using confusion matrices and area ratio of the cumulative gain curve as accuracy metrics, we find that neural networks generally perform better than logistic regression. We also study ways of preprocessing the data, where we find that balancing the data prior to training can give a much better result. On this balanced data, we found the best parameters for the neural network to be a learning rate of $\eta=10^{-1}$, three hidden layers with $80$ nodes each and $10.000$ SGD epochs. This classifier has an accuracy score of $0.86$ and an area ratio of $0.65$ (table \ref{tab:results_classification}), and $0.64$ of the classifications were true positives and $0.92$ true negatives (table \ref{tab:confusion_logreg}). We also study the performance of neural networks when applied to a regression problem, and compare this with the performance of common regression methods. We find that the neural network performs best when using the ReLU function as an activation function, anda learning rate of $\eta=10^{-1}$. The lowest MSE of $0.0415$ is obtained when using three hidden layers with $70$ nodes each, while the Ordinary Least Squares method give an MSE of $0.0136$ when applied to the same data set (table \ref{tab:regression}).
